\section{Algorithm description}

\subsection{Notation}
{\color{red} многие факты здесь введены, наверное, не очень своевременно}

\begin{itemize}
  \item $\M$~--- множество тех молекул, среди которых хотим организвовать поиск
  \item $substructure(M_1, M_2)$~--- предикат для молекул $M_1, M_2 $, который равен $true $ тогда и только тогда, когда $M_1 $ является подстуктурой $M_2$   
  \item Фингерпринт~--- битовая строка константной длины. Зафикисруем константу $fl$ равную длине фингерпринта. 
  \item Для каждой молекулы можно построить фингерпринт. $fp: \M \to \F$~--- функция, которая строит фингерпринт по молекуле.
  \item $F[i]$~--- $i $~-й бит фингерпринта $F$
  \item Будем писать $F_1 \le F_2$, для фингерпринтов $F_1, F_2$ если: $\forall i \in \{1, 2, \ldots, fl\} \ F_1[i] \le F_2[i]$
  \item $\forall M_1, M_2 \in \M  \ substructure(M_1, M_2) \Rightarrow fp(M_1) \le fp(M_2)$ {\color{red} вроде не относится к Notation, но надо где-то зафиксировать. Вероятно, это будет позже обозначено в секции <<About fingerprint-based screening>>}
  \item $\F = \{fp(M) \mid M \in \M\}$~--- множество фингерпринтов построенных по множеству $\M$
  \item $\F_M = \{F' \in \F \mid fp(M) \le F'\}$~--- множество всех фингерпринтов, являющихся надмаской фингерпринта $fp(M)$. 
  \item $\T$~--- бинарное дерево поиска для фингерпринтов из $\F$.
  \item $\T.root$~--- корень дерева 
  \item $d $~--- глубина дерева $\T$. Дерево $\T$ будет полным бинарным деревом глубины $d$
  \item $v.left, v.right$~--- левое и правое соотвественно поддеревья вершины $v$ бинарного дерева
  \item $v.set$~--- множетсво фингерпринтов, хранящееся в вершине $v$ бинарного дерева. При этом $\bigsqcup\limits_{l \text{~--- лист } \T} l.set = \F$ (каждый элемент $\F$ лежит ровно в одном листе)

  \item $v.leaves$~--- множество всех листов поддерева вершины $v$
  \item $v.centroid = \bigvee\limits_{l \in v.leaves}\bigvee\limits_{F \in l.set} F$~--- центроид записанный в вершине $v$.
    По смыслу $v.centroid$~--- это фингерпринт $F$, у которого $\forall i $ $F[i] = 1$ тогда и только тогда, когда существует фингерпринт $F' $ в поддереве $v$, такой что $F'[i] = 1$. {\color{red} centroid~--- термин из BallTree, который выглядит немного неестественно в данном частном случае. Возможно, стоит ввести переобозначение}
\end{itemize}

\subsection{Общая идея}

{\color{red} насколько нужен этот раздел? Ниже всё равно опишем то же самое, но более формально}

\begin{itemize}
  \item Не будем работать напрямую с молекулами. Для заданного множества $\M$ построим множество $\F$ и будем решать задачу о поиске $\{F' \in \F \mid F \text{ is submask of } F'\}$ для заданного фингерпринта $F$ 
\item Тогда для поиска всех надструктур молекулы $M$ сначала найдём множество $\F_M$. Тогда ответом будет $\{ M' \in \bigcup\limits_{F' \in \F_M} fp^{-1}(F') \mid M' \text{ is substructure of } M\}$, где $fp^{-1}(F') = \{M' \in \M \mid fp(M') = F'\}$, а проверка <<$M' \text{ is substructure of } M$>> осуществляется с помощью сторонних алгоритмов.     
  \item Для эффективного поиска $F_M$ будем использовать BallTree с метрикой Russel-Rao для множества битовый строк $\F$. В общем-то, это довольно частный случай BallTree, поэтому будем описывать ниже нашу идею, без привязки к обобщённой версии BallTree. {\color{red} слишком неаккуратно написана связь нашего дерева с  BallTree}
\end{itemize} 

\subsection{Поиск в дереве}

\begin{itemize}
  \item Для полученной молекулы $M $ строим $F = fp(M)$. И для $F $ запускаем поиск в дереве.
  \item Рекурсивно спускаемся в обоих детей, начиная с корня.
  \item Если оказались в врешине $v $ для которой $F \not\le v.centroid$, то прекращаем рекурсивный спуск из $v$. 
  \item Если дошли таким образом до листа $l $ и $F \le l.centroid$, то  добавим в $F_M $ $\{F' \in v.set \mid fp(M) \le F'\}$ (то есть в листе организуем обычный перебор)
\end{itemize}

Псевдокод функции поиска по фингерпринту в дереве описан в \ref{alg:FindInSubtree}. 
Псевдокод функции поиска надструктур заданной молекулы описан в \ref{alg:FindMetaStructures}. 

{\color{red} в какое-то место надо засунуть про то, что такой подход можно распараллелить}

\begin{algorithm}
  \caption{Поиск всех подходящих фингерпринтов в поддереве}\label{alg:FindInSubtree}
  \begin{algorithmic}[1]
    \Require{$v$~--- вершина в дереве. $F $~--- фингерпринт}
    \Ensure{$\{F' \in \bigcup\limits_{l \in v.leaves} l.set \mid F \le F' \}$}
    \Procedure{FindInSubtree}{$v, F$} 
    \If{$F \not\le v.centroid$} \label{alg:FindInSubtree:line:RecursionCut}
      \State \textbf{return} $\varnothing$
    \ElsIf{$v \text{ is leaf}$}
      \State \textbf{return} $\{F' \in v.set \mid F \le F' \}$ 
    \Else
      \State $left \gets $ \Call{FindInSubtree}{$v.left, F$}  
      \State $right \gets $ \Call{FindInSubtree}{$v.right, F$} 
      \State \textbf{return} \Call{Concatenate}{$left, right$} 
    \EndIf
    \EndProcedure
  \end{algorithmic}
\end{algorithm}

\begin{algorithm}
  \caption{Поиск всех надструктур заданной молекулы} \label{alg:FindMetaStructures}
  \begin{algorithmic}[1]
    \Require $M $~--- молекула 
    \Ensure $\{M' \in \M \mid substructure(M, M') \}$ 
    \Procedure{FindMetaStructures}{$M $}
    \State $F \gets fp(M) $ 
    \State $F_M \gets $ \Call{FindInSubtree}{$\T.root, F$}
    \State \textbf{return} $\bigcup\limits_{F' \in F_M} fp^{-1}(F')$ \Comment{$fp^{-1}(F) = \{M' \in \M \mid fp(M) = F\}$} 
    \EndProcedure
  \end{algorithmic}
\end{algorithm}

\subsection{Построение дерева}

\begin{algorithm}
  \caption{Построение дерева} \label{alg:BuildTree}
  \begin{algorithmic}[1]
    \Require $\F$~--- множество фингерпринтов, $d $~--- глубина дерева
    \Ensure $\T $~--- BallTree, для поиска надмасок фингерпринтов 
    \Procedure{BuildTree}{$\F, d$}
      \State $v \gets$ new node
      \If{$d = 1$} \Comment{{\color{red} уточнить, остановка на $d = 1$ или на $d = 0$}} 
	\State $v.set \gets \F$ 
	\State $v.centroid \gets \bigvee\limits_{F \in \F} F$ 
	\State \textbf{return} $v$ 
      \Else 
        \State $\F_l, \F_r \gets $ \Call{SplitFingerprints}{$\F$}
        \State $v.left \gets $ \Call{BuildTree}{$\F_l, d - 1$} 
	\State $v.right \gets $ \Call{BuildTree}{$\F_r, d - 1$}
	\State $v.centroid \gets v.left.centroid \lor v.right.centroid$ 
        \State \textbf{return} $v$ 
      \EndIf
    \EndProcedure
  \end{algorithmic}
\end{algorithm}

Для начала создадим тривиальное дерево из одной вершины $\T.root$. Зададим $\T.root.set = \F$. Далее будем индукционно разделять листы дерева на 2 части и тем самым добавлять новые вершины в дерево.
Более формально, будем для каждого листа $l$ дерева разделять $l.set$  с помощью некоторой функции SplitFingerprints: $\F_l, \F_r \gets \text{SplitFingerprints}(l.set)$ ($\F_l \sqcup \F_r = l.set$).
А далее рекурсивно строить деревья $l.left, l.right$ для множеств $\F_l, \F_r$.

Будем продолжать такое разделение листов до тех пор, пока $\T$ не станет полным бинарным деревом глубины $d$. Псевдокод с описанным выше алгоритмом можно найти в \ref{alg:BuildTree}. 

{ \color{red} Хочется где-то явно обозначить, что $\T.root \gets $ BuildTree($\F, d$) }
% после добавления предыдущей строчки заголовки алгоритмов красятся. 

\begin{algorithm}
  \caption{Алгоритм разделение фингерпринтов попалам при построении дерева} \label{alg:SplitFingerprints}
  \begin{algorithmic}[1]
    \Require $\F$~--- множество фингерпринтов для разделения
    \Ensure $\F_l, \F_r$~--- разделённое множество $\F$  
    \Procedure{SplitFingerprints}{$\F $}
      \State $b \gets \argmin\limits_{i}\{ \left| |\F| - 2k \right| \mid k = \# \{F \in \F \mid F_i = 1 \} \}$ \Comment{{\color{red} стоит ли пояснить формулу?}}
      \State $\F_l \gets \{F \in \F \mid F[i] = 0\}$
      \State $\F_r \gets \{F \in \F \mid F[i] = 1\}$ 
      \If {$|\F_l| > \lfloor \frac{n}{2} \rfloor$}
	\State $\F_r \gets \F_r \ \cup$ \Call{TakeLastElements}{$\F_l, |\F_l| - \lfloor \frac{n}{2} \rfloor$}
	\State $\F_l \gets $ \Call{DropLastElements}{$\F_l, |\F_l| - \lfloor \frac{n}{2} \rfloor$} 
      \ElsIf{$|\F_r| > \lceil \frac{n}{2} \rceil$}
	\State $\F_l \gets \F_l \ \cup$ \Call{TakeLastElements}{$\F_r, |\F_r| - \lceil \frac{n}{2} \rceil$}
	\State $\F_r \gets $ \Call{DropLastElements}{$\F_r, |\F_r| - \lceil \frac{n}{2} \rceil $} 
      \EndIf
      \State \textbf{return} $\F_l, \ \F_r$ 
    \EndProcedure
  \end{algorithmic}
\end{algorithm}

Хотим так выполнять разделения, чтобы в среднем часто происходили отсечения при переборе ветвей. То есть часто выполнялся $if$ в строке \ref{alg:FindInSubtree:line:RecursionCut} алгоритма \ref{alg:FindInSubtree}. Обсудим детальнее работу функции SplitFingerprints.

Изначально мы пробовали выбирать некоторый бит $i$, и отправлять в левое поддерево все фингерпринты $F$, такие что $F[i] = 0$, а в правое поддерево те, у которых $F[i] = 1$. Тогда при поиске надструктур фингерпринта $F'$, если $F'[i] = 1$, то отсекаем всё левое поддерево. 
На практике получается, что после небольшого количества разделений при выборе любого бита левая и правая доли сильно различаются. Поэтому не получается сделать достаточно глубокое сбаланстированное дерево. А дерево малой глубины или несбалансированное дерево не даёт серьёзного прироста по сравнению с полным перебором, так как почти не отсекает ветви перебора. {\color{red} стоит ли описать формальнее идею? Или может вообще надо убрать данное описание, так как в итоговом алгоритме используется другой подход}

Поэтому была выбрана следующая идея: будем выбирать бит так же как описано выше. Но по факту делить так, чтобы размеры совпадали. То есть если лучшее разделение $n $ фингерпринтов даёт доли размеров $n_0, n_1 (n_0 < n_1 \ \land \ n_0 + n_1 = n)$, то в левую долю отправятся все значений с нулём, а значения с единцей распределятся так, чтобы итоговый размер левой и правой долей были равны $\lfloor\frac{n}{2}\rfloor, \lceil \frac{n}{2} \rceil$ соотвественно. Если $n_0 > n_1$, то будем действтивать симмtтрично. Алгоритм функции SplitFingerprints можно найти в псевдокоде \ref{alg:SplitFingerprints}


