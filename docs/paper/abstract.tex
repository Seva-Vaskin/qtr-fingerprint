\section{Introduction}


Достижения в синтетической химии приводят к тому, что молекулы, синтезируемые в настоящее время, состоят из более сложных объектов с механическими связями и более обширных каркасов. Все более важным становится вопрос о том, как пользователи могут эффективно искать такие структуры в больших базах данных.

Задача поиска химических соединений, содержащих заданный фрагмент, в больших базах данных также является одной из задач в процессе разработки лекарств и позволяет решать конкретные проблемы при разработке новых лекарств. Рассмотрим на конкретном примере.

{

\begin{wrapfigure}{l}{0.5\textwidth}
    \includegraphics[width=0.35\textwidth]{img/abstract_example.png}
\end{wrapfigure}


Эта молекула является активной, но не может быть использована по причине:
\begin{itemize}
  \item фармакологических проблем (ADMET - absorption, distribution, metabolism, excretion and toxicity)
  \item защищенности патентом
\end{itemize}

Мы знаем что активность данной молекулы обусловлена выделенным фрагментом.

Значит существует вероятность найти активные молекулы среди ее производных, содержащих данный фрагмент.

}

{\color{red} проврить достоверность примера. Точно ли данная молекула обладает данными проблемами? Может составитель задачи подобрал пример с потолка?

Сделать нормальную картинку в \LaTeX}


\subsection{Постановка задачи}

Рассмотрим химическое соединение $C$:

\begin{center}
\chemfig[angle increment=30,bond join=true]{
  *6(
  (
  -[-5]-[5]-
  {\color{blue}H_2N}
  )
  -=
  *6(
  -
  {\color{red}O}
  ---
  {\color{red}O}
  -
  )
  -=-=
  )
}
\end{center}

Требуется разработать алгоритм, который будет находить все такие химические соединения $C^{'}$, в которых $C$ является подструктурой $C^{'}$:

\begin{figure}[h]
  \label{figure:result}
  \center{\includegraphics[width=1\linewidth]{img/abstract_metastructures.png}}
\end{figure}

{\color{red} сделать картинку в \LaTeX}

