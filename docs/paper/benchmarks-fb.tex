\section{Benchmarks}

In this study, we have carried out comprehensive benchmarking to assess the performance of our algorithm, which is an extension 
of the Bingo fingerprinting system, in comparison with established indices, namely Bingo and Sachem/Lucy. Our benchmarking process 
was performed under the following conditions:

\begin{itemize}
\item OS: Ubuntu 22.04
\item Processor: Intel Xeon E5-2686 v4 (Broadwell)
\item Clock speed: 2.7 GHz
\item RAM: 120 GB
\end{itemize}

The query dataset used for benchmarking was retrieved from \url{https://hg.sr.ht/~dalke/sqc/browse?rev=tip}, which contains 3488 
relevant queries for substructure search. Ten queries were excluded due to various issues, resulting in a final set of 3478 compounds.

We believe that this comparison is appropriate because the Sachem search in the referenced study was conducted in a similar manner. 
However, there are some notable differences between our testing conditions and those of the referenced studies. Our testing was 
performed on a processor with a higher clock speed (2.7 GHz versus 2.6 GHz), the database used in our tests was larger (113M 
compared to 94M), and our search was conducted for the first 10,000 results.

Despite these differences, the benchmark results provide a meaningful comparison of the relative efficiencies of the three systems. 
For a single-threaded in-memory execution, our algorithm demonstrates competitive performance, and it also shows the potential for 
parallelization, exhibiting substantial improvements when executed on 16 threads in memory.

The table below summarizes these benchmark timings, providing a clear comparison between our algorithm, Bingo, and Sachem/Lucy.

\begin{center}
\begin{tabular}{|c|c|c|c|c|}
    \hline
    \% & \begin{tabular}{@{}c@{}}Our Algorithm,\\ single-threaded\\ in-memory\end{tabular} (s) & \begin{tabular}{@{}c@{}}Our Algorithm,\\ 16-threaded\\ in-memory\end{tabular} (s) & Bingo (s) & Sachem/Lucy (s) \\
    \hline
    50\% & 2.17443 & 0.337058 & 1 & - \\
    60\% & 3.83995 & 0.525275 & - & - \\
    70\% & 4.87392 & 0.677609 & - & - \\
    80\% & 6.71327 & 0.895 & 10 & - \\
    85\% & - & - & - & 1 \\
    90\% & 12.9814 & 1.65519 & - & - \\
    95\% & 30.2751 & 3.75599 & 100 & 10 \\
    98\% & - & - & - & 10 \\
    100\% & - & - & - & $<$80 \\
    \hline
\end{tabular}
\end{center}

This thorough analysis offers valuable insights into the performance and potential scalability of our algorithm, especially when it comes to parallel computing.
